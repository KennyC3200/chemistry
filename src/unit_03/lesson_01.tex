\lesson{Introduction to Enthalpy}
The part of the universe one chooses to study is called a \textbf{system}. The three types of
systems are
\footnote{
    \textbf{Thermochemistry:} the study of energy changes that accompany physical or chemical changes
    of matter.
}
\begin{enum}
    \item \textbf{Open system:} one system that can exchange both matter and energy with its
        surroundings
    \item \textbf{Closed system:} one that can exchange energy with its surroundings, but not matter
    \item \textbf{Isolated system:} one that exchanges neither energy nor matter with its
        surroundings
\end{enum}

\subsection{Enthalpy}
Energy transfers can either occur as heat ($q$)
\footnote{
    \textbf{Heat:} the amount of energy transferred between substances.
}
or as work ($w$). Energy transfer affects the total
energy contained within a system; that is, its internal energy ($E$).\\

The components of internal energy that interests us, as Chemists, are:
\begin{bulleted-list}
    \item \textbf{Thermal energy:} energy associated with random molecular motion
        \footnote{
            \textbf{Thermal energy:} energy available from a substance as a result of the motion
            of its molecules.
        }
    \item \textbf{Chemical energy:} energy associated with chemical bonds and intermolecular forces
        \footnote{
            \textbf{Chemical system:} a set of reactants and products under study, usually
            represented by a chemical equation
        }
\end{bulleted-list}
Enthalpy is denoted by $H$ and represents the sum of the internal energy and the pressure-volume
product of the system
\[
    H=E+PV
\]
In a lab context, the pressure is generally constant, thus the change in enthalpy is
\[
    \Delta H=\Delta E+P\Delta V
\]
\textbf{Note:} Since $P\Delta V$ is usually negligible, especially when dealing with solids or
liquids, we can simply say $\Delta H\approx\Delta E$.\\

Thus, $H$ is the heat content of a system and $\Delta H$ is the enthalpy change of a reaction.
To determine the type of reaction, we need to consider the sign of $\Delta H$
\begin{bulleted-list}
    \item For an \textbf{endothermic reaction}, $\Delta H>0$
    \item For an \textbf{exothermic reaction}, $\Delta H<0$
\end{bulleted-list}
The internal energy of a system is known as a \textbf{function state}. This means that the internal
energy, and thus enthalpy, depends on temperature, pressure, kind of substance, and amount of 
substance present.

\subsection{Molar Enthalpy}
The molar enthalpy, denoted by $\Delta H_x$, is the enthalpy change associated with a chemical,
physical, or nuclear change mole of a particular substance.
\[
    \Delta H_x=\frac{\Delta H}{n}
\]
In other words, molar enthalpy is just a rate of enthalpy, with respect to moles.

\subsection{Measuring Energy Change (Enthalpy Change)}
Enthalpy change is the difference in enthalpies between the reactants and products during a change.
For most chemical reactions, it is very difficult to measure the energy change of a substance
through an entire reaction. Most significantly, because the chemicals change into new substances;
therefore, what is present at the beginning is different than what is produced.\\

Therefore, chemists most often measure energy changes \textbf{indirectly} using devices called
calorimeters
\footnote{
    \textbf{Calorimetry:} the technological process of measuring energy changes in a chemical
    system.
}
and calculating the energy change of the surroundings to infer the energy change of
the chemical system.
\[
    \Delta H_\text{system}=-q_\text{surroundings}
\]
This relationship exists because the energy lost by the system will be gained by the surroundings,
and vice versa. This is via the conservation of energy.\\

The three different types of enthalpy changes are
\begin{enum}
    \item \textbf{Physical change:} a change in the form of a substance, in which no chemical
        bonds are broken
    \item \textbf{Chemical change:} a change in the chemical bonds between atoms, resulting in the
        arrangement of atoms into new substances
    \item \textbf{Nuclear change:} a change in the protons or neutrons in an atom, resulting in
        the formation of new atoms
\end{enum}

\subsection{Calculating Energy Change}
The heat energy can be calculated by using the equation
\[
    q=mc\Delta T
\]
Recall that $m$ is the mass, $c$ is the specific heat capacity, and $\Delta T$ is the change in
temperature, in degrees celsius.

\begin{table}[!ht]
    \scriptsize
    \centering
    \caption{Molar enthalpies for changes in states of selected substances}
    \setlength{\tabcolsep}{10pt}      % column spacing
    \renewcommand{\arraystretch}{1.2} % row spacing
    \arrayrulecolor{black}            % table border color
    \begin{tabular}{|c|c|c|c|}
        \hline
        \rowcolor{HeaderColor}
        Chemical formula & Formula & Molar enthalpy of fusion (kJ/mol) & Molar enthalpy of evaporation (kJ/mol) \\ \hline
        sodium & $\ch{Na}$ & 2.6 & 101 \\ \hline
        chlorine & $\ch{Cl2}$ & 6.4 & 20.4 \\ \hline
        sodium chloride & $\ch{NaCl}$ & 28 & 171 \\ \hline
        water & $\ch{H2O}$ & 6.03 & 40.8 \\ \hline
        ammonia & $\ch{NH3}$ & - & 1.37 \\ \hline
        freon-12 & $\ch{CCl2F2}$ & - & 34.99 \\ \hline
        methanol & $\ch{CH3OH}$ & - & 39.23 \\ \hline
        ethylene glycol & $\ch{C2H4(OH)2}$ & - & 58.8 \\ \hline
    \end{tabular}
\end{table}

\begin{sample}{A common refrigerant, Freon-12, molar mass 120.91 g/mol, is alternately vaporized
    in test tubes inside a refrigerator, absorbing heat, and condensed in tubes outside the
    refrigerator, releasing heat. This results in energy being transferred from the inside to
    the outside of the refrigerator. The molar enthalpy of vaporized for the refrigerant is
    34.99 kJ/mol. If 500.0 g of the refrigerant is vaporized, what is the expected enthalpy
    change $\Delta H$?}
    Solve for the number of moles of Freon-12
    \begin{align*}
        n&=\frac{500.0\,\si{g}}{120.91\,\si{g.mol^{-1}}}\\
         &=4.13497\,\si{mol}
    \end{align*}
    Multiply the number of moles by the molar enthalpy
    \begin{align*}
        \Delta H&=(4.13497\,\si{mol})(34.99\,\si{kJ.mol^{-1}})\\
                &=144.6826\,\si{kJ}
    \end{align*}
\end{sample}

\begin{sample}{What amount of ethylene glycol would vaporize while absorbing 200.0 kJ of heat?}
    The enthalpy of vaporization of ethylene glycol is 58.8 \si{kJ.mol^{-1}}. Using the formula
    for enthalpy heat, we can solve for $n$
    \begin{align*}
        \Delta H_\text{vap}&=\frac{\Delta H}{n}\\
        n&=\frac{\Delta H}{\Delta H_\text{vap}}\\
         &=\frac{200.0\,\si{kJ}}{58.8\,\si{kJ.mol^{-1}}}\\
         &=3.40\,\si{mol}
    \end{align*}
    Therefore, the amount of ethylene glycol evaporated is 3.40 mol.
\end{sample}

\begin{sample}{Calculate the enthalpy change for the vaporization of 100.0 g of water at 100.0 \celsius.}
    From the formula $\Delta H_x=\frac{\Delta H}{n}$, $\Delta H=n\Delta H_x$, all we need to solve
    for is $n$ then multiply it by $\Delta H_x$
    \begin{align*}
        n&=\frac{100.0\,\si{g}}{16.00\,\si{g.mol^{-1}}+2(1.01\,\si{g.mol^{-1}})}\\
         &=5.54938957\,\si{mol}\\
        \Delta H&=(5.54938957\,\si{mol})(40.8\,\si{kJ.mol^{-1}})\\
                &=226.415\,\si{kJ}
    \end{align*}
\end{sample}

\begin{sample}{Ethylene glycol is used in automobile coolant systems because its aqueous solutions
    lower the freezing point of the coolant liquid to prevent freezing of the system during
    Canadian winters. What is the enthalpy change needed to completely vaporize 500.0 g of ethylene
    glycol?}
    Using the formula for molar enthalpy and solving for enthalpy change $\Delta H=n\Delta H_\text{vap}$
    \begin{align*}
        n&=\frac{500.0\,\si{g}}{2(12.01\,\si{g.mol^{-1}})+6(1.01\,\si{g.mol^{-1}})+2(16.00\,\si{g.mol^{-1}})}\\
         &=8.05412\,\si{mol}\\
        \Delta H&=n\Delta H_\text{vap}\\
                &=(8.05412\,\si{mol})(58.8\,\si{kJ.mol^{-1}})\\
                &=473.58\,\si{kJ}
    \end{align*}
\end{sample}

\subsection{Calorimetry of Physical Changes}
Studying energy change requires an isolated system. Two nested disposable polystyrene cups are
a fairly effective calorimeter. There are three simplifying assumptions often used in calorimetery:
\begin{enum}
    \item No heat is transferred between the calorimeter and the outside environment
    \item Any heat absorbed or released by the calorimeter materials, such as the container, is
        negligible
    \item A dillute aqeuous solution is assumed to have a density and specific heat capacity equal
        to that of pure water (1.00 g/mL and 4.18 J/(g\celsius))
\end{enum}
\textbf{Note:} what this section boils down to is using specific heat capacity and enthalpy change
to calculate the molar enthalpy, and vice versa.

\subsection{Calorimetry of Chemical Changes}
Chemical reactions are also studied using calorimeters. This setup usually involves aqeuous 
reactant solutions that are considered to be equivalent to water (recall the three rules). Thus,
the procedure is the same as the ones in the caorimetry of physical changes. When aqueous 
solutions of acids and bases react, they neutralize, while producing enthalpy.
