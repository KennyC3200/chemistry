\lesson{Introduction to Enthalpy}
The part of the universe one chooses to study is called a \textbf{system}. The three types of
systems are
\begin{enum}
    \item \textbf{Open system:} one system that can exchange both matter and energy with its
        surroundings
    \item \textbf{Closed system:} one that can exchange energy with its surroundings, but not matter
    \item \textbf{Isolated system:} one that exchanges neither energy nor matter with its
        surroundings
\end{enum}

\subsection{Enthalpy}
Energy transfers can either occur as heat ($q$) or as work ($w$). Energy transfer affects the total
energy contained within a system; that is, its internal energy ($E$).\\

The components of internal energy that interests us, as Chemists, are:
\begin{bulleted-list}
    \item \textbf{Thermal energy:} energy associated with random molecular motion
    \item \textbf{Chemical energy:} energy associated with chemical bonds and intermolecular forces
\end{bulleted-list}
Enthalpy is denoted by $H$ and represents the sum of the internal energy and the pressure-volume
product of the system
\[
    H=E+PV
\]
In a lab context, the pressure is generally constant, thus the change in enthalpy is
\[
    \Delta H=\Delta E+P\Delta V
\]
\textbf{Note:} Since $P\Delta V$ is usually negligible, especially when dealing with solids or
liquids, we can simply say $\Delta H\approx\Delta E$.\\

Thus, $H$ is the heat content of a system and $\Delta H$ is the enthalpy change of a reaction.
To determine the type of reaction, we need to consider the sign of $\Delta H$
\begin{bulleted-list}
    \item For an \textbf{endothermic reaction}, $\Delta H>0$
    \item For an \textbf{exothermic reaction}, $\Delta H<0$
\end{bulleted-list}
The internal energy of a system is known as a \textbf{function state}. This means that the internal
energy, and thus enthalpy, depends on temperature, pressure, kind of substance, and amount of 
substance present.

\subsection{Molar Enthalpy}
The molar enthalpy, denoted by $\Delta H_x$, is the enthalpy change associated with a chemical,
physical, or nuclear change mole of a particular substance.
\[
    \Delta H_x=\frac{\Delta H}{n}
\]
In other words, molar enthalpy is just a rate of enthalpy, with respect to moles.

\subsection{Measuring Energy Change}
For most chemical reactions, it is very difficult to measure the energy change of a substance
through an entire reaction. Most significantly, because the chemicals change into new substances;
therefore, what is present at the beginning is different than what is produced.\\

Therefore, chemists most often measure energy changes \textbf{indirectly} using devices called
calorimeters and calculating the energy change of the surroundings to infer the energy change of
the chemical system.
\[
    \Delta H_\text{system}=-q_\text{surroundings}
\]
This relationship exists because the energy lost by the system will be gained by the surroundings,
and vice versa. This is via the conservation of energy.

\subsection{Calculating Energy Change}
The heat energy can be calculated by using the equation
\[
    q=mc\Delta T
\]
Recall that $m$ is the mass, $c$ is the specific heat capacity, and $\Delta T$ is the change in
temperature, in degrees celsius.

\begin{table}[!ht]
    \scriptsize
    \centering
    \caption{Molar enthalpies for changes in states of selected substances}
    \setlength{\tabcolsep}{10pt}      % column spacing
    \renewcommand{\arraystretch}{1.2} % row spacing
    \arrayrulecolor{black}            % table border color
    \begin{tabular}{|c|c|c|c|}
        \hline
        \rowcolor{HeaderColor}
        Chemical formula & Formula & Molar enthalpy of fusion (kJ/mol) & Molar enthalpy of evaporation (kJ/mol) \\ \hline
        sodium & $\ch{Na}$ & 2.6 & 101 \\ \hline
        chlorine & $\ch{Cl2}$ & 6.4 & 20.4 \\ \hline
        sodium chloride & $\ch{NaCl}$ & 28 & 171 \\ \hline
        water & $\ch{H2O}$ & 6.03 & 40.8 \\ \hline
        ammonia & $\ch{NH3}$ & - & 1.37 \\ \hline
        freon-12 & $\ch{CCl2F2}$ & - & 34.99 \\ \hline
        methanol & $\ch{CH3OH}$ & - & 39.23 \\ \hline
        ethylene glycol & $\ch{C2H4(OH)2}$ & - & 58.8 \\ \hline
    \end{tabular}
\end{table}

\begin{sample}{A common refrigerant, Freon-12, molar mass 120.91 g/mol, is alternately vaporized
    in test tubes inside a refrigerator, absorbing heat, and condensed in tubes outside the
    refrigerator, releasing heat. This results in energy being transferred from the inside to
    the outside of the refrigerator. The molar enthalpy of vaporized for the refrigerant is
    34.99 kJ/mol. If 500.0 g of the refrigerant is vaporized, what is the expected enthalpy
    change $\Delta H$?}
    Solve for the number of moles of Freon-12
    \begin{align*}
        n&=\frac{500.0\,\si{g}}{120.91\,\si{g.mol^{-1}}}\\
         &=4.13497\,\si{mol}
    \end{align*}
    Multiply the number of moles by the molar enthalpy
    \begin{align*}
        \Delta H&=(4.13497\,\si{mol})(34.99\,\si{kJ.mol^{-1}})\\
                &=144.6826\,\si{kJ}
    \end{align*}
\end{sample}

\begin{sample}{What amount of ethylene glycol would vaporize while absorbing 200.0 kJ of heat?}
    The enthalpy of vaporization of ethylene glycol is 58.8 \si{kJ.mol^{-1}}. Using the formula
    for enthalpy heat, we can solve for $n$
    \begin{align*}
        \Delta H_\text{vap}&=\frac{\Delta H}{n}\\
        n&=\frac{\Delta H}{\Delta H_\text{vap}}\\
         &=\frac{200.0\,\si{kJ}}{58.8\,\si{kJ.mol^{-1}}}\\
         &=3.40\,\si{mol}
    \end{align*}
    Therefore, the amount of ethylene glycol evaporated is 3.40 mol.
\end{sample}

\begin{sample}{Calculate the enthalpy change for the vaporization of 100.0 g of water at 100.0 }
\end{sample}
