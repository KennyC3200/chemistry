\lesson{Hess's Law and Summation Method}
\textbf{Hess's Law} states that the value of $\Delta H$ for any reaction that can be written in
steps equals the sum of the values of $\Delta H$ for each of the individual steps.
\begin{align*}
    \Delta H_\text{target}&=\Sigma\Delta H_\text{known}\\
                          &=\Delta H_1+\Delta H_2+\Delta H_3+\dots
\end{align*}

We can use Hess's Law to estimate the enthalpy change of a large reaction, that may be dangerous
for instance, into smaller reactions and estimate the enthalpy change of the large reaction.\\

The summation method is a method that utilizes Hess's Law to calculate the enthalpy change of a
reaction as the sum of the enthalpy change's of smaller reactions
\[
    \Delta H^\circ_\text{r}=\Sigma\Delta H^\circ_\text{f}\text{ products}-\Sigma\Delta H^\circ_\text{f}\text{ reactants}
\]
\textbf{Note:} this method must use enthalpies of formation, $\Delta H_f$.

% Molecule project
% \newpage
% \begin{center}
%     \vspace{2em}
%     \setchemfig{chemfig style={line width=1pt}}
%     \chemfig{*6(=-(-[:-30](=[:-90]O))=-(-[:90]O-[:30])=(-[:150]HO)-)}
%
%     \vspace{1em}
%     Vanillin ($\ch{C8H8O3}$)\\
% \end{center}
%
% \begin{center}
%     \vspace{2em}
%     \setchemfig{chemfig style={line width=1pt}}
%     \chemfig{*6(=(-[:-90]-[:210](-[:150]H_2N)-[:-90](-[:210]HO)=[:-30]O)-=-(-[:90]O-[:30]P(=[:-30]O)(-[:60]OH)(-[:120]HO))=-)}
%
%     \vspace{1em}
%     O-Phospho-L-tyrosine ($\ch{C9H12NO6P}$)
% \end{center}
%
% \begin{center}
%     \vspace{2em}
%     \setchemfig{chemfig style={line width=1pt}}
%     \chemfig{*6(=-(-[:-30](=[:-90]O))=(-[:30]O-[:-30])-(-[:90]O-[:30]P(=[:-30]O)(-[:60]OH)(-[:120]HO))=(-[:150]~[:150])-)}
%
%     \vspace{1em}
%     Tinsu ($\ch{C10H9O6P}$)\\
% \end{center}
%
% \begin{center}
%     \vspace{2em}
%     \setchemfig{chemfig style={line width=1pt}}
%     \chemfig{*6(=-(-[:-30](=[:-90]O)-[:30]OH)=(-[:30]O-[:-30])-(-[:90]O-[:30]P(=[:-30]O)(-[:60]OH)(-[:120]HO))=(-[:150]~[:150])-)}
% \end{center}
