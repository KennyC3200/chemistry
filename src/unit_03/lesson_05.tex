\lesson{Bond Energy Method}
Bond energy and bond length are related; the greater the bond energy, the shorter the bond. 
\begin{bulleted-list}
    \item Energy is \textbf{released} when atoms join together to form covalent bonds
    \item Energy is \textbf{absorbed} to break apart covalently bonded atoms
    \item \textbf{Bond-dissociation energy} ($D$) is the quantity of energy required to break one
        mole of covalent bonds in a gaseous species (kJ/mol)
\end{bulleted-list}

\begin{sample}{Calculate the bond dissociation energy in water}
    We can consider one mole of water as bonds between $\ch{H_{(g)}}$ and $\ch{OH_{(g)}}$; H-OH. 
    Then, the $\ch{OH_{(g)}}$ will be have a bond considered as O-H.
    \begin{align*}
        \ch{H-OH_{(g)}}\to \ch{H_{(g)}}+\ch{OH_{(g)}}\quad&\Delta H=D(\ch{H-OH})=+498.7\,\si{kJ/mol}\\
        \ch{OH_{(g)}}\to \ch{H_{(g)}}+\ch{O_{(g)}}\quad&\Delta H=D(\ch{O-H})=+428.0\,\si{kJ/mol}
    \end{align*}
\end{sample}

\begin{important}
    The bond dissociation energy provided in a table is not the actual value, but rather the average.
    An \textbf{average bond energy} is the average of the bond-dissociation energies for a number
    of different species containing one particular bond. Therefore, using them to estimate 
    enthalpy change is not as accurate as with Hess's Law and the Summation Method.
\end{important}

The overall enthalpy change of a reaction can be done by using either of the equations below
\begin{align*}
    \Delta H_\text{rxn}&=\Delta H\text{(bond breakage)}+\Delta H\text{(bond formation)}\\
    \Delta H_\text{rxn}&=\Delta\text{BE(reactants)}-\Delta\text{BE(products)}
\end{align*}
