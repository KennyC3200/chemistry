\lesson{The Solubility Product Constant}
Solid ionic compounds dissociate to form anions and cations in an aqueous solution
\[
    \ch{CaSO4_{(s)}}\rightleftharpoons \ch{Ca^{2+}_{(aq)}}+\ch{SO4^{2-}_{(aq)}}
\]
When the solid salt is initially added to water, there are no calcium or sulfate ions present.
However, as the reaction proceeds, the concentration of ions begins to increase. Therefore, it
becomes more likely that they ions will collide with each other and reform the solid phase
\begin{align*}
    \ch{CaSO4_{(s)}}&\rightleftharpoons \ch{Ca^{2+}_{(aq)}}+\ch{SO4^{2-}_{(aq)}}\\
    \ch{Ca^{2+}_{(aq)}}+\ch{SO4^{2-}_{(aq)}}&\rightleftharpoons \ch{CaSO4_{(s)}}
\end{align*}
Eventually, in a closed system, equlibrium is reached
\[
    \ch{Ca^{2+}_{(aq)}}+\ch{SO4^{2-}_{(aq)}}\rightleftharpoons \ch{CaSO4_{(s)}}
\]
At equilibrium, it would appear that no more solids are dissolving because the solution is
\textbf{saturated}. However, the reality is that the forward reaction rate equals the reverse
reaction rate. The equilibrium constant for this specific type of reaction is the \textbf{solubility
product constant}, denoted by $K_\mathrm{sp}$.

\begin{important}
    Excess solids must be present in the saturated solution in order to establish this equilibrium.
\end{important}

In our example, the solubility product constant would be 
\[
    K_\mathrm{sp}=[\ch{Ca^{2+}_{(aq)}}][\ch{SO4^{2-}_{(aq)}}]
\]
Note that there is no denominator in this case because the reactants are solids.

\subsection{Solubility Product vs. Solubility}
\begin{bulleted-list}
    \item The \textbf{solubility product} for a given reaction is an \textit{equilibrium constant}.
        Therefore, it has only one value at a specific temperature
    \item The term \textbf{solubility} refers to the amount of a particular substance that is
        able to dissolve in solution, measured as a concentration
\end{bulleted-list}

\begin{important}
    Determining the solubility is useful for ensuring that no reactants are wasted, since there
    is no point adding reactants beyond the maximum saturation of the solution.
\end{important}

\begin{sample}{copper(I) bromide has a measured solubility of $2.0\times10^{-4}\,\si{mol.L^{-1}}$
    at 25$^{\circ}$C. Calculate the solubility product constant.}
    The BCE is
    \[
        \ch{CuBr_{(s)}}\rightleftharpoons \ch{Cu^{1+}_{(aq)}}+\ch{Br^{1-}_{(aq)}}
    \]
    Therefore, the concentration of $\ch{Cu^{1+}_{(aq)}}$ and $\ch{Br^{1-}_{(aq)}}$ ions
    are the same as the solubility of $\ch{CuBr_{(aq)}}$
    \begin{align*}
        K_\mathrm{sp}&=[\ch{Cu^{1+}_{(aq)}}][\ch{Br^{1-}_{(aq)}}]\\
                     &=(2.0\times10^{-4})(2.0\times10^{-4})\\
                     &=4.0\times10^{-8}
    \end{align*}
\end{sample}

\begin{sample}{Calculate the solubility product constant for bismuth sulfide, $\ch{Bi2S3}$, if it
    has a measured solubility of $1.0\times10^{-15}\,\si{mol.L^{-1}}$ at 25$^{\circ}$C.}
    The BCE is
    \[
        \ch{Bi2S3_{(s)}}\rightleftharpoons \ch{2Bi^{3+}_{(aq)}}+\ch{3S^{2+}_{(aq)}}
    \]
    \begin{align*}
        K_\mathrm{sp}&=[\ch{Bi^{3+}_{(aq)}}]^2[\ch{S^{2+}_{(aq)}}]^3\\
                     &=(2.00\times10^{-15})^2(3.00\times10^{-15})3\\
                     &=1.08\times10^{-73}
    \end{align*}
\end{sample}

\begin{sample}{The solubility product constant for zinc sulfide is $6.1\times10^{-24}$ at 25$^{\circ}$C.
    Determine the individual ion concentrations.}
    The BCE is
    \[
        \ch{ZnS_{(s)}}\rightleftharpoons \ch{Zn^{2+}_{(aq)}}+\ch{S^{2+}_{(aq)}}
    \]
    Since we do not know the concentrations of the ions, let them both be denoted $x$, since they
    have the same coefficients
    \begin{align*}
        K_\mathrm{sp}&=[\ch{Zn^{2+}_{(aq)}}][\ch{S^{2+}_{(aq)}}]\\
        1.6\times10^{-24}&=(x)(x)\\
        x^2&=1.6\times10^{-24}\\
        x&=1.26\times10^{-12}\,\si{mol.L^{-1}}
    \end{align*}
\end{sample}

\begin{sample}{The $K_\mathrm{sp}$ for copper(II) iodate, $\ch{Cu(IO3)2}$, is $1.4\times10^{-7}$
    at 25$^{\circ}$C. Calculate its solubility at this temperature.}
    The BCE is 
    \[
        \ch{Cu(IO3)2_{(s)}}\rightleftharpoons \ch{Cu^{2+}_{(aq)}}+\ch{2IO3^-_{(aq)}}
    \]
    Similar to the last sample, we let $x$ equal the solubility of $\ch{Cu(IO3)2}$
    \begin{align*}
        K_\mathrm{sp}&=[\ch{Cu^{2+}_{(aq)}}][\ch{IO3^-_{(aq)}}]^2\\
        1.4\times10^{-7}&=(x)(2x)^2\\
        x&=(\frac{1.4\times10^{-7}}{4})^\frac{1}{3}\\
         &=3.24\times10^{-3}\,\si{mol.L^{-1}}
    \end{align*}
\end{sample}

\subsection{Precipitation}
The $K_\mathrm{sp}$ value can also be used to predict whether or not a precipitate will form when
two solutions are mixed. Calculate the reaction quotient, $Q$, and compare it to $K_\mathrm{sp}$.
\begin{bulleted-list}
    \item If $Q<K_\mathrm{sp}$, the solution is unsaturated; therefore, \textbf{no precipitate}
    \item If $Q>K_\mathrm{sp}$, the solution is saturated; therefore, there \textbf{is precipitate}
    \item If $Q=K_\mathrm{sp}$, the solution is exactly at the saturation point; therefore, \textbf{no precipitate}
\end{bulleted-list}

\begin{sample}{Will a precipitate form if 25.0 mL of 0.00400 M aluminum hydroxide solution 
    mixed with 25.0 mL of 0.0150 M calcium nitrate solution? $K_\mathrm{sp}$ for $\ch{Ca(OH)2}$
    is $6.5\times10^{-6}$.}
    Note that this is a double displacement reaction, which is why we are given the $K_\mathrm{sp}$
    for $\ch{Ca(OH)2}$. To determine if a precipitate will form, we need to calculate for $Q$ and
    compare it with $K_\mathrm{sp}$. To calculate $Q$, we need to determine $[\ch{Ca^{2+}_{(aq)}}]$
    and $[\ch{OH^-_{(aq)}}]$. One important thing to keep in mind is that the combined volume of
    the two solutions is 50 mL. For $\ch{Al(OH)3}$
    \begin{align*}
        C_1V_1&=C_2V_2\\
        C_2&=\frac{C_1V_1}{V_2}\\
           &=\frac{(0.004\,\si{M})(25\,\si{mL})}{50\,\si{mL}}\\
           &=0.002\,\si{M}
    \end{align*}
    For $\ch{Ca(NO3)2}$
    \begin{align*}
        C_1V_1&=C_2V_2\\
        C_2&=\frac{C_1V_1}{V_2}\\
           &=\frac{(0.015\,\si{M})(25\,\si{mL})}{50\,\si{mL}}\\
           &=0.0075\,\si{M}
    \end{align*}
    Therefore, the concentrations we need are
    \begin{align*}
        [\ch{OH^-_{(aq)}}]&=3[\ch{Al(OH)3}]=3(0.002\,\si{M})=0.006\,\si{M}\\
        [\ch{Ca^{2+}_{(aq)}}]&=[\ch{Ca(NO3)2}]=0.0075\,\si{M}
    \end{align*}
    The BCE for the dissociation reaction is
    \[
        \ch{Ca(OH)2_{(s)}}\rightleftharpoons \ch{Ca^{2+}_{(aq)}}+\ch{2OH^-_{(aq)}}
    \]
    Therefore, the $Q$ value is
    \begin{align*}
        Q&=[\ch{Ca^{2+}_{(aq)}}][\ch{OH^-_{(aq)}}]^2\\
         &=(0.0075)(0.06)^2\\
         &=2.70\times10^{-7}
    \end{align*}
    Since $Q<K_\mathrm{sp}$, the solution is unsaturated, and no precipitate is formed.
\end{sample}
