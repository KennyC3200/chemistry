\lesson{Acid and Base Dissociation Constants}
Water undergoes \textbf{autoionization}. Roughly, there are 1 in a billion successful collisions in pure 
water at SATP.
\begin{align*}
    K_w&=1.4\times10^{-14}\\
    [\ch{H3O^+_{(aq)}}]&=1.0\times10^{-7}\\
    [\ch{OH^-_{(aq)}}]&=1.0\times10^{-7}\\
\end{align*}

Similar to the solubility product constant, the acid and base dissociation constants are calculated
via writing the balanced dissociation equation and plugging in the concentrations.\\

Keep in mind that strong acids have a percent reaction >99\%, thus we do not consider equilibrium.
For a weak acid however, because their percent reaction <99\%, we can consider their equilibrium.
For a weak acid $\ch{HA_{(aq)}}$
\[
    \ch{HA_{(aq)}}+\ch{H2O_{(\ell)}}\rightleftharpoons \ch{H3O^+_{(aq)}}+\ch{A^-_{(aq)}}
\]
The acid dissociation constant is
\[
    K_a=\frac{[\ch{H3O^+_{(aq)}}][\ch{A^-_{(aq)}}]}{\ch{HA_{(aq)}}}
\]

\begin{important}
    Fluoric acid, $\ch{HF_{(aq)}}$, is actually a \textbf{strong acid}. This is because fluorine
    is extremely electronegative.
\end{important}

Keep in mind that for polyprotic acids, they actually undergo a series of equilibrium reactions.
Consider for instance the weak acid $\ch{H3PO4_{(aq)}}$, which is triprotic and actually undergoes 
three different equilibrium reactions simultaneously, which each successive reaction become
weaker than the previous.
\begin{align*}
    \ch{H3PO4_{(aq)}}+\ch{H2O_{(\ell)}}\rightleftharpoons \ch{H3O^+_{(aq)}}+\ch{H2PO4^-_{(aq)}}\\
    \ch{H2PO4^-_{(aq)}}+\ch{H2O_{(\ell)}}\rightleftharpoons \ch{H3O^+_{(aq)}}+\ch{HPO4^{2-}_{(aq)}}\\
    \ch{H2PO4^{2-}_{(aq)}}+\ch{H2O_{(\ell)}}\rightleftharpoons \ch{H3O^+_{(aq)}}+\ch{PO4^-_{(aq)}}\\
\end{align*}
In most cases, the latter 2 reactions are negligible since the acid gets progressively weaker. 
Furthermore, there is a $\ch{H3O^+_{(aq)}}$ \textbf{common ion effect}, promoting the reverse 
reaction even more. Thus, in polyprotic acids, we mainly only consider the first equilibrium reaction.
In this case, the acid dissociation constant is
\[
    K_a=\frac{[\ch{H3O^+_{(aq)}}][\ch{H2PO4^-_{(aq)}}]}{\ch{H3PO4_{(aq)}}}
\]
The base dissociation is exactly the same as the acid dissociation constant. Given a weak base
$\ch{B_{(aq)}}$
\[
    \ch{B_{(aq)}}+\ch{H2O_{(\ell)}}\rightleftharpoons \ch{BH^+_{(aq)}}+\ch{OH^-_{(aq)}}
\]
The dissociation constant is
\[
    K_b=\frac{[\ch{BH^+_{(aq)}}][\ch{OH^-_{(aq)}}]}{[\ch{B_{(aq)}}]}
\]
\begin{important}
    Given $K_a$ and $K_b$, their product equals $K_w$. Keep in mind that $a$ and $b$ should be
    the acid and conjugate base, or conjugate acid and base, respectively.
    \[
        (K_a)(K_b)=K_w
    \]
    For instance, consider acetic acid
    \begin{align*}
        (K_a)(K_b)&=\left(\frac{[\ch{H3O^+_{(aq)}}][\ch{CH3COO^-_{(aq)}}]}{\ch{CH3COOH_{(aq)}}}\right)\left(\frac{[\ch{CH3COOH_{(aq)}}][\ch{OH^-_{(aq)}}]}{[\ch{CH3COO^-_{(aq)}}]}\right)\\
                  &=[\ch{H3O^+_{(aq)}}][\ch{OH^-_{(aq)}}]\\
                  &=K_w
    \end{align*}
\end{important}
