\lesson{The Reaction Quotient}
\subsection{Equilibrium Establishment}
\begin{bulleted-list}
    \item When the reactants and products for a particular chemical reaction are combined, it
        is useful to know whether is mixture is at equilibrium or not. When it is not, it is
        necessary to know how the system will change in order to establish equilibrium
    \item This change in equilibrium is typically referred to as a \textit{shift}
    \item A shift \textbf{to the right} means that the concentration of the \textbf{products} will
        increase to reach equilibrium
    \item A shift \textbf{to the left} means that the concentration of the \textbf{reactants} will
        increase to reach equilibrium
\end{bulleted-list}

Determining the \textit{shift} in position is very simple if the concentration of any one of the
reactants or products is 0:
\begin{bulleted-list}
    \item The system must shift in the direction of the missing component(s)
    \item This is because it is impossible to reach equilibrium if one of the components is missing
\end{bulleted-list}

\subsection{The Reaction Quotient}
\begin{bulleted-list}
    \item When all of the initial concentrations are non-zero, then the reaction quotient, denoted
        by $Q$, is used to determine the shift that is required to establish equilibrium
    \item The reaction quotient uses the equilibrium constant expression for the reaction;
        however, the \textbf{initial concentrations} are used in the equation, rather than the
        equilibrium concentrations
\end{bulleted-list}
For instance, in the reaction $\ch{N2_{(g)}}+3 \ch{H2_{(g)}}\rightleftharpoons2 \ch{NH3_{(g)}}$ 
\[
    Q=\frac{[\ch{NH3}]^2_o}{[\ch{N2}]_o[\ch{H2}]^3_o}
\]
Where the subscripted $o$ denotes that it is the \textbf{initial concentration}.\\

Recalling that $Q$ and $K$ measure the ratio between the concentrations of the products to reactants,
we can compare the values of $Q$ to $K$ to assert the following
\begin{bulleted-list}
    \item When $Q>K$
        \begin{bulleted-list}
            \item The initial concentration ratio between products to reactants is \textbf{too large}
            \item Some products will be consumed to form reactants in equilibrium
            \item The system shifts \textbf{left} to reach a state of equilibrium
        \end{bulleted-list}
    \item When $Q<K$
        \begin{bulleted-list}
            \item The initial concentration ratio between products to reactants is \textbf{too little}
            \item Some reactants will be consumed to form products in equilibrium
            \item The system shift \textbf{right} to reach a state of equilibrium
        \end{bulleted-list}
    \item When $Q=K$
        \begin{bulleted-list}
            \item The initial concentration represents a system that is already in a state of
                equilibrium
            \item The system will undergo no shift
        \end{bulleted-list}
\end{bulleted-list}

\begin{sample}{For the synthesis of ammonia at $500^{\circ}$C, the equilibrium constant is $K=6.0\times10^{-2}$.
    Predict the direction that the system must shift in order to establish equilibrium for the
    following ``initial concentration'' scenarios.
    \begin{table}[!ht]
        \footnotesize
        \centering
        \setlength{\tabcolsep}{12pt}      % column spacing
        \renewcommand{\arraystretch}{1.2} % row spacing
        \arrayrulecolor{black}            % table border color
        \begin{tabular}{|c|c|c|c|}
            \hline
            Experiment & [$\ch{N2}$] & [$\ch{H2}$] & [$\ch{NH3}$] \\ \hline
            1 & $1.0\times10^{-5}$ & $2.0\times10^{-3}$ & $1.0\times10^{-3}$ \\ \hline
            2 & $1.5\times10^{-5}$ & $3.54\times10^{-1}$ & $2.0\times10^{-4}$ \\ \hline
            3 & $5.0$ & $1.0\times10^{-2}$ & $1.0\times10^{-4}$ \\ \hline
        \end{tabular}
    \end{table}
}
    For experiment 1, the value for $Q$ is
    \begin{align*}
        Q&=\frac{[\ch{NH3}]^2}{[\ch{N2}][\ch{H2}]^3}\\
         &=\frac{(1.0\times10^{-3})^2}{(1.0\times10^{-5})(2.0\times10^{-3})^3}\\
         &=1.25\times10^{7}
    \end{align*}
    Since $Q>K$, the reaction shifts to the left.\\
    For experiment 2, the value for $Q$ is
    \begin{align*}
        Q&=\frac{[\ch{NH3}]^2}{[\ch{N2}][\ch{H2}]^3}\\
         &=\frac{(2.0\times10^{-4})^2}{(1.5\times10^{-5})(3.54\times10^{-1})^3}\\
         &=0.06
    \end{align*}
    Since $Q\approx K$, the reaction has no shift.\\
    For experiment, 3, the value for $Q$ is
    \begin{align*}
        Q&=\frac{[\ch{NH3}]^2}{[\ch{N2}][\ch{H2}]^3}\\
         &=\frac{(1.0\times10^{-4})^2}{(5.0)(1.0\times10^{-2})^3}\\
         &=0.002
    \end{align*}
    Since $Q<K$, the reaction will shift right.
\end{sample}
