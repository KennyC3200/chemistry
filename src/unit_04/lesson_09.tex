\lesson{Buffers}
The most common application of acid-base solutions are \textbf{buffered solutions}. A buffer
is a solution that resists change in pH when either hydroxide ions or protons are added
\begin{bulleted-list}
    \item A buffered solution may contain a \textbf{weak acid and a salt containing its conjugate
        base} (e.x. $\ch{HF}$ and $\ch{NaF}$) or a \textbf{weak base and a salt containing its 
        conjugate acid} (e.x. $\ch{NH3}$ and $\ch{NH4Cl}$)
    \item By choosing the correct components, a solution can be buffered to almost any pH
    \item A buffer is able to resist significant changes in pH because it has both
        the acid and basic components. These components neutralize added $\ch{H^+}$ or $\ch{OH^-}$
\end{bulleted-list}

When the concentrations of the conjugate acid/base is the same as the base/acid, the target pH
of the solution is simply the $\mathrm{pK_a}$. 
\begin{align*}
    K_a&=\frac{[\ch{H3O^+}][\ch{A^-}]}{[\ch{HA}]}\\
       &=[\ch{H3O^+}]\\
    \mathrm{pK_a}&=\mathrm{pH}
\end{align*}
This occurs when $[\ch{A^-}]=[\ch{HA}]$. At this point, this is known as the \textbf{half neutralization
point}, where half of the volume of the equivalence point is required.

\begin{important}
    The half neutralization point allows us to utilize $\mathrm{pK_a}=\mathrm{pH}$ which allows us to
    identify the acid.
\end{important}

\subsection{How Does Buffering Work?}
Suppose a buffer contains a relatively large amount of weak acid $\ch{HA}$
\[
    \ch{HA_{(aq)}}+\ch{H2O_{(\ell)}}\rightleftharpoons \ch{A^-_{(aq)}}+\ch{H3O^+_{(aq)}}
\]
When hydroxide ions are added to the solution, the weak acid is the best source of protons
\[
    \ch{OH^-_{(aq)}}+\ch{HA_{(aq)}}\to \ch{H2O_{(\ell)}}+\ch{A^-_{(aq)}}
\]
The net result is that $\ch{OH^-_{(aq)}}$ ions do not build up in the solution, rather they are
replaced by $\ch{A^-_{(aq)}}$ ions. Recall that the solution already contains $\ch{A^-_{(aq)}}$
ions, from the salt. Recall that
\begin{align*}
    K_a&=\frac{[\ch{H3O^+_{(aq)}}][\ch{A^-_{(aq)}}]}{[\ch{HA_{(aq)}}]}\\
    [\ch{H3O^+_{(aq)}}]&=K_a\frac{[\ch{HA_{(aq)}}]}{[\ch{A^-_{(aq)}}]}
\end{align*}
If the original concentrations of [$\ch{HA}$] and [$\ch{A-}$] are initially very large, when
$\ch{OH^-}$ ions are added, the change in the ratio $[\ch{HA}]$:$[\ch{A^-}]$ will be minimal. 

\subsection{Identifying Strength of Buffers}
\begin{bulleted-list}
    \item \textbf{Acidic buffers:} made from weak acids and their conjugate bases (the acid dominates)
    \item \textbf{Basic buffers:} made from weak bases and their conjugate acids (the base dominates)
    \item \textbf{Near neutral buffers:} have balanced weak acids and bases
\end{bulleted-list}
