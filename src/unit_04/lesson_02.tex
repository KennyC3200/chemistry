\lesson{Equilibrium Constant}
A ratio involving equilibrium constants between the reactants and products has a constant value,
independent of how the equilibrium is reached. The ratio can be derived using the BCE. This ratio
is called the equilibrium constant expression and denoted by $K_\text{eq}$.\\

In the hypothetical reaction
\[
    aA+bB\rightleftharpoons cC+dD
\]
The value for $K_\text{eq}$, determined \textbf{empirically}, is given mathematically as
\[
    K=\frac{[\mathrm{C}]^c[\mathrm{D}]^d}{[\mathrm{A}]^a[\mathrm{B}]^b}
\]
Where the exponents are the same as the coefficients in the BCE.
\begin{bulleted-list}
    \item When calculating the equilibrium coditions for homogeneous equilibrium, the expression
        is very straight forward. However, many equilibria involve more than one phase and are
        called heterogeneous
    \item \textbf{Homogeneous equilibria:} equilibria in which all entities are in the same phase
    \item \textbf{Heterogeneous equilibria:} equilibria in which reactants and products are
        in more than one phase
    \item When writing the equilibrium constant expression, it is essential to note that the
        position of the equilibrium \textbf{does not depend on the amounts of solid of liquid
        components present}. This is because, fundamentally, the ``concentrations'' of pure solids
        and liquids do not change. Therefore, the ``concentrations'' of these components are not
        included in the equilibrium constant expression
    \item In other words, since solids and liquids generally do not change concentrations, they are
        \textbf{not included} in the equilibrium constant equation
\end{bulleted-list}

\subsection{Derivation of the Equilibrium Constant}
Given equilibrium equation
\[
    \mathrm{A+B} \xrightleftharpoons[v_f]{v_r} \mathrm{C+D}
\]
Where $v_f$ is the rate of the forward reaction and $v_r$ is the rate of the reverse reaction,
we can derive two equations
\begin{align*}
    v_f&=k_f[\mathrm{A}][\mathrm{B}]\\
    v_r&=k_r[\mathrm{C}][\mathrm{D}]
\end{align*}
During equilibrium, the forward reaction rate is equivalent to the reverse reaction rate. That is,
$v_f=v_r$
\begin{align*}
    v_f&=v_r\\
    k_f[\mathrm{A}][\mathrm{B}]&=k_r[\mathrm{C}][\mathrm{D}]\\
    \frac{k_f}{k_r}&=\frac{[\mathrm{C}][\mathrm{D}]}{[\mathrm{A}][\mathrm{B}]}\\
    K&=\frac{[\mathrm{C}][\mathrm{D}]}{[\mathrm{A}][\mathrm{B}]}
\end{align*}
Of course, this applies to the case where the reactants have different coefficients as well
\[
    K=\frac{[\mathrm{C}]^c[\mathrm{D}]^d}{[\mathrm{A}]^a[\mathrm{B}]^b}
\]
