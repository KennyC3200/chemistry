\lesson{Different Types of Radioactive Emissions}
There are five main types of emissions: alpha emission, beta emission, positron emission, electron 
capture, and gamma emission. Four of these produce changes in the elements undergoing decay, and 
the end result is a more stable atomic structure.

\subsection{Alpha Emission}
These emissions result in the release of an alpha particle from the atom. Recall that an 
a-particle is a helium nucleus. The result in alpha decay is the atom’s atomic number decreasing 
by two and the mass number decreasing by four. An example of an a-decay is:
\[
    \ch{^{238}_{92}U}\to \ch{^{4}_{2}H}+\ch{^{234}_{90}Th}
\]

\subsection{Beta Emission}
Although there are two types of b-particles (b1 and b2), the former is usually referred to as a 
positron, so we’ll refer to only the b2 particle as a beta particle. In a beta emission, a beta 
particle is ejected from the atom. A beta particle has all of the properties of an electron 
(virtually massless, negative charge), yet it is created by the conversion of a neutron in the 
nucleus to a proton and an electron (beta particle). The proton remains in the nucleus, and the 
beta particle is ejected from the nucleus. An example of a beta emission is:
\[
    \ch{^{227}_{89}Ac}\to \ch{^{0}_{-1}e}+\ch{^{227}_{90}Th}
\]
Notice that since the number of nucleons in the atom does not change, the mass number remains 
unchanged. However, the gain of a proton increases the atomic number by one (and consequently 
changes the element).

\subsection{Position Emission}
Positron emissions are also known as b1 emissions. The positron is known as an antiparticle. 
Antiparticles are the exact opposites of particles. The positron is the antiparticle to an 
electron and is represented by the symbol +1 0 e. The electron has virtually no mass and a charge 
of 21 (relative to a proton). A positron has virtually no mass and a charge of 11. When an 
electron and its antiparticle, the positron, collide, they disintegrate and their matter is 
converted entirely into energy in the form of two gamma rays. In a positron emission, a proton in 
the nucleus is converted into a neutron and a positron. The neutron remains in the nucleus, and 
the positron is ejected. The life span of the positron is very brief since it will disintegrate 
upon collision with an electron. An example of a positron emission can be seen in the example 
showing the breakdown of carbon-11:
\[
    \ch{^{11}_{6}C}\to \ch{^{0}_{+1}e}+\ch{^{11}_{5}B}
\]
Notice that the number of nucleons doesn’t change here, either. As a result, the mass number of 
the atom does not change. However, the conversion of a proton to a neutron decreases the atomic 
number by one (and changes the element).

\subsection{Electron Capture}
The fourth type of emission is called electron capture. In this process, an inner shell electron 
is pulled into the nucleus, and when this occurs, the electron combines with a proton to form a 
neutron. In electron capture reactions, the atomic number decreases by one, the mass number 
remains the same, and the element changes. One difference in this type of reaction is that the 
electron is written to the left of the arrow to show that it is consumed, rather than produced, 
in the process. An example of electron capture can be seen in the following reaction:
\[
    \ch{^{201}_{80}Hg}+\ch{^{0}_{-1}e}\to \ch{^{201}_{79}Au}
\]
