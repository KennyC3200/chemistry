\lesson{Valence Bond Theory}
\begin{bulleted-list}
    \item A half-filled orbital in one atom can overlap with another half-filled orbital of a
        second atom to form a new, bonding orbital
    \item When atoms bond, they arrange themselves in space to achieve the maximum overlap 
        of their half-filled orbitals. Maximum overlap produces a bonding orbital of lowest energy
\end{bulleted-list}

\lesson{VSEPR Theory}
\textbf{VSEPR Theory} stands for Valence Shell Electron Pair Repulsion Theory; pairs of electrons
in the valence shell of an atom stay as far apart as possible to minimize the repulsion of their
negative charges. There are two forms of geometry:
\begin{enum}
    \item Electron-pair geometry
    \item Molecular geometry
\end{enum}
The electron-pair geometry and molecular geometry are sometimes the same ($\ch{CH4}$ for example),
but the majority of the time they are different ($\ch{NH3}$ for example)

\begin{important}
    In VSEPR bonding
    \begin{bulleted-list}
        \item Only the valence shell electrons of the central atoms are important for molecular
            shape
        \item The molecular shape is determined by the positions of the electrons when they are a 
            maximum distance apart to yield the lowest repulsion possible
    \end{bulleted-list}
\end{important}

\subsection{VSEPR Notation}
VSEPR notation is denoted as AXE, where
\begin{bulleted-list}
    \item A represents the number of central atoms (always just 1)
    \item X represents the number of atoms bonded to the central atom
    \item E represents the number of electron pairs on the central atom
\end{bulleted-list}
