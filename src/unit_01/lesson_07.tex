\lesson{Lewis Theory of Bonding}
\begin{bulleted-list}
    \item \textbf{Ionic bonding:} the electrostatic attraction between positive and negative ions
        in the crystal lattice of a salt
    \item \textbf{Covalent bonding:} the sharing of valence electrons between atomic nuclei
        within a molecule or complex ion
    \item \textbf{Valence:} the number of unpaired electrons in the outermost energy level
    \item The Dalton atom theory starts before the Mendeleev periodic table with Franklind stating
        that each element has a fixed valence that determines its bonding capacity
    \item Friedrich Keule extended the idea of illustrating a bond as a dash between bonding atoms;
        i.e. what we now call structural diagram
    \item Jacbobus van't Hoff and Joseph Le Bel extended these structures to three dimensions
    \item Gilbert Lewis combined the knowledge of many chemical formulas, the concept of valence,
        the octet rule, and the electron-shell model of the atom to explain chemical bonding
    \item The electrons that do not participate in bonding but move between bonded molecules are
        said to be \textbf{delocalized electrons}
\end{bulleted-list}

\begin{important}
    The key ideas of Lewis theory of bonding are:
    \begin{bulleted-list}
        \item Atoms and ions are stable if they have a noble gas-like electron structure; i.e.
            a stable octet of electrons
        \item Electrons are most stable when they are paired
        \item Atoms form chemical bonds to achieve a stable octet of electrons
        \item A stable octet may be achieved by an exchange of electrons between metal and nonmetal
            atoms
    \end{bulleted-list}
\end{important}

\begin{sample}{Place the following chemistry concepts in the order that they were created by
    chemists:
    \begin{enum-alph}
        \item Lewis structures
        \item Empirical formnulas
        \item Dalton atom
        \item Kekule structures
        \item Schrodinger quantum mechanics
    \end{enum-alph}
}
    (b), (c), (d), (e), (f)
\end{sample}

\begin{sample}{Place the following chemistry concepts in the order that they were created by
    chemists and briefly explain how one concept led to the next:
    \begin{enum-alph}
        \item Bohr atom
        \item Empirical formulas
        \item Dalton atom
        \item Lewis structures
    \end{enum-alph}
}
    The order is empirical formulas, Dalton atom, Bohr atom, and Lewis structures. Empirical
    formulas were developed to express the simplest whole-number ratio of elements in a compound.
    Dalton built on the concept of empirical formulas by suggesting that atoms of different 
    elements combined in fixed ratios form compounds. Bohr's model built on Dalton's model.
    Lastly, Lewis structures built on Bohr's understanding of electron orbits--note that the
    whole idea of Lewis structures is built on the valence shell of the electron, which implies
    Bohr's atom
\end{sample}
