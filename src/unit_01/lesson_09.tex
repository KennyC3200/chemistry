\lesson{Intermolecular Forces}
\textbf{Intermolecular forces} are forces of attraction that exist \textbf{BETWEEN} molecules.\\
\textbf{Intramolecular forces} are forces of attraction that exist \textbf{WITHIN} a molecule.
\begin{important}
    Bonds do NOT BREAK/BROKEN in changes of state. Instead, you should say: more energy is required
    to overcome the \textbf{intermolecular attraction} between the molecules. We are simply separating
    molecules.
\end{important}

\subsection{Interionic Forces}
Generally, the attractive force between a pair of oppositely charged ions is greater the larger the
charge on the ions with a decrease in ionic size. For ionic compounds, the intramolecular forces
are equivalent to the intermolecular forces, since they are both electrostatic forces.

\subsection{London Dispersion Forces}
These forces involve the displacement of elcetrons in molecules. They result from the \textbf{random motion}
of electrons and occur when the concentration of electrons is higher in one particular region than
another. This results in what is called a \textbf{instantaneous dipole}. Once the electrons ``realize''
that there are too many in one particular area, they \textbf{disperse}.
\begin{bulleted-list}
    \item The instantaneous dipole from one atom can also affect the neighbouring molecules. For instance,
        if a molecule on the right had a partial negative charge on the left, then the molecule
        on the left will have a partial repulsion force on the right
    \item This effect creates an \textbf{induced dipole}, creating a force of attraction between
        the two atoms
    \item These induced dipoles of course have a chain-effect, where one induced-dipole creates
        another induced-dipole
    \item I.e. the random motion of electron creates induced dipoles. This is also why nonpolar
        substances have lower melting/boiling points, and can be liquid, soft solid, and gas.
\end{bulleted-list}
\begin{important}
    A dipole is more easily induced as the number of electrons increases and in turn as molar mass increases.
    As atomic radius increases, so does induction. Because dispersion forces become stronger as
    molar mass increasees, melting/boiling point of covalent compounds also increases with molar
    mass. For instance, helium has a BP of 4 K and radon has a BP of 211 K.
\end{important}

\subsection{Dipole-Dipole Interactions}
In a polar substance, molecules tend to line up the ``positive'' end of one dipole with the
``negative'' end of neighbouring dipoles. This type of interaction requires a polar substance;
therefore, it is often called a \textbf{permanent dipole}. This yields higher melting/boiling
points.\\

\begin{important}
    When comparing substances of roughly the same molar mass, the dipole forces will produce significant
    differences in melting/boiling point. However, when comparing substances of widely different molar
    masses, the dispersion forces are usually more significant than dipole forces. That is, as
    molar mass increases, the london forces are stronger than the dipole forces.
\end{important}

\subsection{Hydrogen Bonding}
A form of dipole-dipole interaction, this is primarily only seen in atoms that have a highly
electronegative atom, for instance N, O, or F. This way, the proton in the nucleus of the hydrogen
is revealed and the bonding force becomes extremely evident. For water to have such a high boiling
point, it is due to the hydrogen bonding, since oxygen is extremely electronegative.
